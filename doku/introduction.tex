\chapter{Einführung}
\label{chap:introduction}
Dieses Dokument dient zum besseren Verständnis unseres Projekts respektive Fallbeispiels im Modul \enquote{Java Soft. Entwicklung mit Open Source 1}. 


\section{Idee}
Wir wollten als Projekt ein DVD-Verleihsystem umsetzen. Dabei soll es Usern möglich sein, eigene DVDs zu erfassen, welche man verleihen möchte. Diese DVDs lassen sich einem Film zuordnen. Andere User wiederum können diese DVDs ausleihen und schliesslich, wenn sie den Film gesehen haben, ein Review (Beurteilung) verfassen. Diese wird wiederum im System dargestellt. Falls von einem Film gerade keine DVD frei verfügbar ist, kann der User eine reservieren. 

Der Nutzen ist, dass User einen Überblick über die Filmsammlung haben und andererseits neue Streifen entdecken könne, die sie noch nicht besitzen. Dank des Reviews können sie sich bereits ein Bild darüber machen, was andere über den Film denken und ob sich eine Ausleihe lohnt.

\section{Methodik}
Das Exempel diente nicht nur zur Adaption der im Unterricht behandelten Themen, sondern auch, um die Arbeit und Kommunikation im Team zu schulen. Zuerst wurde ein Domain Model erstellt, welches aber laufen angepasst wurde. Auch die spätere Implementation erfuhr mehrere Lebenszyklen. So wurde nicht nur eine im Unterricht erarbeitete Lösung abgeben, sondern man traf sich auch noch ausserschulisch und kann schliesslich eine Umsetzung mit SpringBoot vorweisen.
