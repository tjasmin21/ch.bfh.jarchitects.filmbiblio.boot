\chapter{Ziel}
\label{chap:ziel}

\section{Hauptziel}
Das Hauptziel ist, dass nach Beendigung des Projektes ein Machbarkeitsnachweis für ein Game bereitgestellt ist, welches Mehrspielerfähig ist, wobei mobile Devices per Browser als Spielcontroller fungieren.

Weitere Informationen finden Sie im Kapitel \ref{chap:ziel} auf Seite \pageref{chap:ziel}.


\begingroup
\setlength{\tabcolsep}{10pt} % Default value: 6pt
\renewcommand{\arraystretch}{1.5}
\section{Unterziele}
\begin{tabularx}{\textwidth}{ |l|X| }
  \hline
  \textbf{Z 1}	& \textbf{Wirtschaftliche Ziele}	\\	\hline
  Z 1.1	& Der Entwicklungsaufwand für ein umgesetztes System ist bekannt.  	\\	\hline
  Z 1.2	& Die Betriebskosten für ein umgesetztes System ist bekannt.  	\\	\hline
  Z 1.3	& Es liegt eine Beurteilung des Marktpotentials vor.  	\\	\hline
  \hline
  \textbf{Z 2}	& \textbf{Technische Ziele}	\\	\hline
  Z 2.1	& Es ist eine Hardware evaluiert, die für das System hinreichend ist.  	\\	\hline
  Z 2.2	& Es existiert ein Software Prototyp, der die Nutzlast testen lässt.  	\\	\hline
  Z 2.3	& Die Eignung möglicher Frameworks für die Umsetzung wurde geprüft.	\\	\hline
  \hline
  \textbf{Z 3}	& \textbf{Konzeptionelle Ziele}	\\	\hline
  Z 3.1	& Es existieren graphische Entwürfe für den Einsatz des Systems. 	\\	\hline 
  
  \hline
  \textbf{Z 4}	& \textbf{Ziele aus Anforderungsmanagement}	\\	\hline
  Z 4.1	& Der Systemkontext ist abgegrenzt.	\\	\hline
  Z 4.2	& Die Stakeholder sind definiert.	\\	\hline
  Z 4.3	& Die funktionalen Anforderungen an ein System sind ermittelt.	\\	\hline
  Z 4.4	& Die Qualitätsanforderungen an ein derartiges System sind festgelegt.	\\	\hline
  Z 4.5	& Allfällige Randbedingungen sind evaluiert. 	\\	\hline
\end{tabularx}
\endgroup
